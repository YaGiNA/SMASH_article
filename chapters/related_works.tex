\section{関連研究}
フェイクニュースの検出や真偽分類は、対象をスパム\cite{shen2017discovering}や風評\cite{7023340}、そして虚偽広告\cite{Huang:2017:DFO:3041021.3054233}を含めると新しいトピックではない。
これまでの研究\cite{Shu:2017:FND:3137597.3137600,Ruchansky:2017:CHD:3132847.3132877,Wang:2018:EEA:3219819.3219903}に倣い、意図的に作成され、明確に誤りであると確認できるニュースをフェイクニュースと定義する。

\subsection{フェイクニュース検出}
ニュース記事がもつ情報からフェイクニュースを検出する手法は多く提案されている。
文字情報からは、フェイクニュースが独自の書かれ方をする上に感情的な表現を多用することから、文章のスタイル\cite{DBLP:journals/corr/PotthastKRBS17}や感情的表現の頻度\cite{DBLP:journals/corr/abs-1903-01728}を考慮する手法がある。
また、ディープニューラルネットワーク(DNN)によって検出性能が改善された報告\cite{wang-2017-liar,karimi-tang-2019-learning,karimi-etal-2018-multi}も多い。

ソーシャルコンテキストを考慮した手法も多く提案されており、扱うコンテキストの種類によってユーザベース\cite{Castillo:2011:ICT:1963405.1963500,8397048,DBLP:journals/corr/abs-1904-13355}
・投稿ベース\cite{Yang2019UnsupervisedFN,Tacchini2017SomeLI,Jin:2016:NVE:3016100.3016318}
・ネットワークベース\cite{Wu:2018:TFF:3159652.3159677,DBLP:journals/corr/abs-1902-06673}の3種類に分けられる。

ソーシャルコンテキストの共通した問題点として、ソーシャルコンテキストはユーザの拡散によって生まれる情報であるため早期検出に向かない点が挙げられる。
早期検出の実現へ、TCNN-URGという2層の畳み込みニューラルネットワークとCVAEによるユーザレスポンス生成器を組み合わせたモデルも提案されている\cite{ijcai2018-533}。
ニュース記事を畳み込みニューラルネットワークで特徴化してから隠れ変数を算出し、寄せられたコメントとして尤もらしい単語群を生成することで検出性能が改善されることが報告されている。
しかしながら、TCNN-URGはあくまで尤もらしい単語を生成することに限られ、実際のコメントそのものは生成していない。

\subsection{フェイクニュース生成}
\label{subsec:generate}
自然言語生成タスクの1つとして、架空のニュース記事を作成するGroverモデルがある\cite{NIPS2019_9106}。
このモデルはニュース記事データセットから記事をドメイン・著者・投稿日時・見出し・本文の5要素に分け、無作為に歯抜けにさせた記事の残り部分から歯抜け部分を予測させることで訓練している。
興味深い点として、Groverモデルで生成した記事の方が実在の記事よりも読者が信じる傾向であることが報告されていたことがある。
本研究ではこのモデルを拡張することで、より自然なコメントを生成することを目指した。
