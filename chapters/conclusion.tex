\section{結論}
本研究では、フェイクニュースの早期発見における問題点の解決を試みた。
我々は、ユーザのコメントはニュース記事を評価する際で重要な情報をもたらすものの、
ニュース拡散の初期段階ではコメントが少ない点に着目する。
そこで、Groverモデルを拡張したニューラルネットワークモデルを作成し、分類に有用なコメントを生成することを提案する。
提案モデルのコメント生成による早期発見性能を評価するために、実際のニュースとそれに寄せられたコメントを対象に実験を行った。
その結果、コメントを生成するプロセスが、ファクトチェックによって真偽を判定する際に役立つ可能性が示唆されている。