\section{実験}
\subsection{単語生成の傾向}
\label{subsec:trend}
まずはじめに、正しいニュースとフェイクニュースの記事とコメントのセットから生成されたコメントの傾向を調べた。
いずれのニュース記事とコメントのセットはFakeNewsNetデータセット\cite{Shu2018FakeNewsNetAD}から取得した。
このデータセットは北米で英文記事を対象にファクトチェックを行う団体であるPolitiFact(政治ニュース中心)とGossipCop(芸能ニュース中心)の判断結果を元に正しいニュースとフェイクニュースのラベルが付けられている。
我々はまず実験手法に合わせるために記事に対して最低3件以上コメントが寄せられているセットに対して無作為に3件選出した。
また、PolitiFactから真偽で各200セットを用意して学習を行った。
実際に生成されたコメントに対して、すべてアルファベットを小文字にした上で単語ごとの出現回数と出現確率を算出した。
また、算出するにあたって記号(クォーテーションやピリオド、コンマなど)やURLの削除を行ったほか、``a''や``is''といったストップワードはNLTK\cite{bird-loper-2004-nltk}が提供するメソッドを使用して除外した。
なお、コメントの収集元がTwitterであることから、Twitter独自の用法をもつ記号(ハッシュタグ\#やメンション@)、またコロンは例外として除外しなかった。
以上の処理を行った結果、以下の特徴が得られた。

\begin{itemize}
    \item 真偽問わず最も頻度が高い単語は``via''であり、真偽全体の単語のうち約1.5\%を占めた。
    \item それに続いて``trump''と``obama''が続いたが、いずれも割合は1\%を下回った。
\end{itemize}

また、真偽における傾向差として以下の違いが見られた。

\begin{itemize}
    \item ``via''は真偽単独で見てもそれぞれで最も高い頻度で生成されていた。
    \item フェイクにおける``via''の生成頻度は正しい場合に比べて約2倍であり、その差は約0.9ポイントとフェイク頻出上位10単語中最多だった。
    \item ``breaking:''という単語が``via''に次いで2番目に頻度の差が高い単語であり、その差は0.7ポイントだった。
\end{itemize}

\subsection{検出における生成コメントの影響}
生成コメントの有無によって真偽分類の結果に影響が出るか調べた。
ベースラインとして2つの入力データを用意した。
1つは生成されたコメントを入力せず、記事と実際に投稿された2件のコメントから分類させた場合、
もう1つは実際に投稿されたコメントも入力せず、記事のみから分類させた場合であった。
この実験では、PolitiFactでは十分な量の学習を行うにはセット数が少なかったため、GossipCopから真偽で各2000セットを用意して行った。
実験結果は表\ref{tbl:classify_results}の通りである。
提案モデルは再現率において全体ベストとなったものの、精度においては生成モデルを使わない方が優秀であることが読み取れる。
また、生成されたコメントは共通して文法面にさらなる改善の必要性が残された。

\begin{table}[!t]
    \renewcommand{\arraystretch}{1.3}
    \caption{分類成績}
    \label{tbl:classify_results}
    \centering
    \begin{tabular}{lccc}
        \hline
        入力データ           & 精度 & 再現率 & F値 \\ \hline
        記事本文のみ         & 0.647     & 0.615  & 0.631    \\
        + 実在コメント2件  & \textbf{0.682}     & 0.750  & \textbf{0.714}    \\
        + 生成コメント1件 & 0.590     & \textbf{0.790}  & 0.675    \\ \hline
    \end{tabular}
\end{table}
